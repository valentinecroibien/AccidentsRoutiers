\documentclass[french,]{compterendu}
\usepackage{lmodern}
\usepackage{amssymb,amsmath}
\usepackage{ifxetex,ifluatex}
\usepackage{fixltx2e} % provides \textsubscript
\ifnum 0\ifxetex 1\fi\ifluatex 1\fi=0 % if pdftex
  \usepackage[T1]{fontenc}
  \usepackage[utf8]{inputenc}
\else % if luatex or xelatex
  \ifxetex
    \usepackage{mathspec}
  \else
    \usepackage{fontspec}
  \fi
  \defaultfontfeatures{Ligatures=TeX,Scale=MatchLowercase}
\fi
% use upquote if available, for straight quotes in verbatim environments
\IfFileExists{upquote.sty}{\usepackage{upquote}}{}
% use microtype if available
\IfFileExists{microtype.sty}{%
\usepackage{microtype}
\UseMicrotypeSet[protrusion]{basicmath} % disable protrusion for tt fonts
}{}
\usepackage[left=1.5cm,right=1.5cm,top=1.5cm,bottom=2cm]{geometry}
\usepackage{hyperref}
\hypersetup{unicode=true,
            pdftitle={Facteurs influençant la gravité des accidents corporels routiers},
            pdfauthor={Hugo CARLIN; Valentine CROIBIEN; Emilie PIERQUIN; Adrien SAGRAFENA},
            pdfborder={0 0 0},
            breaklinks=true}
\urlstyle{same}  % don't use monospace font for urls
\ifnum 0\ifxetex 1\fi\ifluatex 1\fi=0 % if pdftex
  \usepackage[shorthands=off,main=french]{babel}
\else
  \usepackage{polyglossia}
  \setmainlanguage[]{}
\fi
% Adding environment CSLReferences for compatibility with pandoc >= 2.8
% BEGIN
\newlength{\cslhangindent}
\setlength{\cslhangindent}{1.5em}
\newenvironment{CSLReferences}%
  {\setlength{\parindent}{0pt}%
  \everypar{\setlength{\hangindent}{\cslhangindent}}\ignorespaces}%
  {\par}
\newenvironment{cslreferences}%
  {\setlength{\parindent}{0pt}%
  \everypar{\setlength{\hangindent}{\cslhangindent}}\ignorespaces}%
  {\par}
% END
\usepackage{longtable,booktabs}
\IfFileExists{parskip.sty}{%
\usepackage{parskip}
}{% else
\setlength{\parindent}{0pt}
\setlength{\parskip}{6pt plus 2pt minus 1pt}
}
\setlength{\emergencystretch}{3em}  % prevent overfull lines
\providecommand{\tightlist}{%
  \setlength{\itemsep}{0pt}\setlength{\parskip}{0pt}}
\setcounter{secnumdepth}{5}
% Redefines (sub)paragraphs to behave more like sections
\ifx\paragraph\undefined\else
\let\oldparagraph\paragraph
\renewcommand{\paragraph}[1]{\oldparagraph{#1}\mbox{}}
\fi
\ifx\subparagraph\undefined\else
\let\oldsubparagraph\subparagraph
\renewcommand{\subparagraph}[1]{\oldsubparagraph{#1}\mbox{}}
\fi

%%% Use protect on footnotes to avoid problems with footnotes in titles
\let\rmarkdownfootnote\footnote%
\def\footnote{\protect\rmarkdownfootnote}



%Mise en page
\usepackage[left=1.5cm,right=1.5cm,top=1.5cm,bottom=2cm]{geometry}
\usepackage{lastpage} %Pour numérotaion des pages
\usepackage{eso-pic} %pour l'image de fond de la page de garde
\usepackage{enumitem} %Pour personnaliser les listes à puces
\usepackage{fancyhdr}
\usepackage{xcolor}

%Gestion des tableaux
\usepackage{multirow}

%Divers
\usepackage{ifthen} %Gestion des instructions conditionnelles


\widowpenalty=10000
\clubpenalty=10000


%%%%%%%%%%%%%%%%%%%%%%%%%%%%%%%%%%%%%%%%%%%%%%%
%% Configuration des messages de type badbox %%
%%%%%%%%%%%%%%%%%%%%%%%%%%%%%%%%%%%%%%%%%%%%%%%

\widowpenalty=10000
\clubpenalty=10000


%%%%%%%%%%%%%%%%%%%%%%%%%%%%%%%
%% Definition environemments %%
%%%%%%%%%%%%%%%%%%%%%%%%%%%%%%%

%%%%%%%%%%%%%%%%%%%%%%%%%%%%%%%%%%%%%%%%%%%%
%% Désignation des variables de la classe %%
%%%%%%%%%%%%%%%%%%%%%%%%%%%%%%%%%%%%%%%%%%%%

  \title{Facteurs influençant la gravité des accidents corporels routiers}
    \author{Hugo CARLIN \\ Valentine CROIBIEN \\ Emilie PIERQUIN \\ Adrien SAGRAFENA}
      \date{21 novembre 2021}


  \email{\href{mailto:hugo.carlin@etudiant.univ-reims.fr}{\nolinkurl{hugo.carlin@etudiant.univ-reims.fr}} \\ \href{mailto:valentine.croibien@etudiant.univ-reims.fr}{\nolinkurl{valentine.croibien@etudiant.univ-reims.fr}} \\ \href{mailto:emilie.pierquin@etudiant.univ-reims.fr}{\nolinkurl{emilie.pierquin@etudiant.univ-reims.fr}} \\ \href{mailto:adrien.sagrafena@univ-reims.fr}{\nolinkurl{adrien.sagrafena@univ-reims.fr}}}
%\email{\href{mailto:hugo.carlin@etudiant.univ-reims.fr}{\nolinkurl{hugo.carlin@etudiant.univ-reims.fr}}\href{mailto:valentine.croibien@etudiant.univ-reims.fr}{\nolinkurl{valentine.croibien@etudiant.univ-reims.fr}}\href{mailto:emilie.pierquin@etudiant.univ-reims.fr}{\nolinkurl{emilie.pierquin@etudiant.univ-reims.fr}}\href{mailto:adrien.sagrafena@univ-reims.fr}{\nolinkurl{adrien.sagrafena@univ-reims.fr}}}
\logouniv{logo_URCA.pdf}
\logoufr{logoUFR.pdf}
\date{21 novembre 2021}
\diplome{Master 2 - Statistique pour l'évaluation et la prévision}
\anac{2020-2021}
\module{SEP0953}
\enseig{Morgan COUSIN}
\evaluation{Compte-rendu d'analyse}


%%%%%%%%%%%%%%%%%%%
%% Mise en forme %%
%%%%%%%%%%%%%%%%%%%


%Formatage en-têtes et pieds de pages
\pagestyle{fancy}
\fancyhead[L]{\small \thetitle}
\fancyhead[R]{}
\fancyfoot[l]{\small \theauthor}
\fancyfoot[C]{\small \it \theeval \ \themodule \ -- \theanac}
\fancyfoot[R]{\small \thepage\ / \pageref{LastPage}}
\renewcommand{\headrulewidth}{0.4pt}
\renewcommand{\footrulewidth}{0.4pt}

\fancypagestyle{plain}{%
\fancyhf{} % clear all header and footer fields
\fancyfoot[C]{\small \thepage\ / \pageref{LastPage}} % except the center
\renewcommand{\headrulewidth}{0pt}
\renewcommand{\footrulewidth}{0pt}}

\AtEndDocument{\thispagestyle{plain}}
% Pandoc header


\begin{document}

\AddToShipoutPictureBG*{\includegraphics[width=\paperwidth,height=\paperheight]{FondURCA.png}}



\maketitle

\pagebreak




{
\hypersetup{linkcolor=black}
\setcounter{tocdepth}{2}
\tableofcontents
}



\begin{verbatim}
## 
## Attaching package: 'dplyr'
\end{verbatim}

\begin{verbatim}
## The following objects are masked from 'package:stats':
## 
##     filter, lag
\end{verbatim}

\begin{verbatim}
## The following objects are masked from 'package:base':
## 
##     intersect, setdiff, setequal, union
\end{verbatim}

\hypertarget{introduction}{%
\section{Introduction}\label{introduction}}

\hypertarget{jeux-de-donnuxe9es-utilisuxe9s}{%
\subsection{Jeux de données utilisés}\label{jeux-de-donnuxe9es-utilisuxe9s}}

Les jeux de données utilisés sont disponibles sur \href{https://www.data.gouv.fr/fr/datasets/bases-de-donnees-annuelles-des-accidents-corporels-de-la-circulation-routiere-annees-de-2005-a-2019/}{DataGouv}. Ces jeux de données décrivent les accidents corporels routiers ayant eu lieu en 2019.

Nous disposons de quatre bases de données :\\
* Base caractéristiques : correspondant aux caractéristiques de l'accident\\
* Base lieux : correspondant aux caractéristiques du lieu de l'accident\\
* Base véhicule : correspondant aux caractéristiques du véhicule prenant part à l'accident\\
* Base usagers : correspondant aux caractéristiques des usagers faisant partie de l'accident

\hypertarget{pruxe9paration-des-donnuxe9es}{%
\section{Préparation des données}\label{pruxe9paration-des-donnuxe9es}}

\hypertarget{jointure-des-4-bases}{%
\subsection{Jointure des 4 bases}\label{jointure-des-4-bases}}

Nous réalisons des jointures entre les différentes bases sur l'identifiant de l'accident. Ainsi nous n'avons plus qu'une base à étudier et non quatre.

\hypertarget{choix-des-variables}{%
\subsection{Choix des variables}\label{choix-des-variables}}

Cependant, en réalisant notre jointure, nous obtenons une base de données avec une multitude de variables.\\
Nous choisissons donc de réaliser un premier tri sur ces variables et gardons les variables suivantes :

\begin{itemize}
\tightlist
\item
  \textbf{Base caractéristiques} :

  \begin{itemize}
  \item
    Num\_acc : Numéro d'identifiant de l'accident
  \item
    Jour mois : Jour et mois de l'accident
  \item
    Hrmn : Heures et minutes de l'accident
  \item
    Lum : Conditions d'éclairage dans lesquelles l'accident s'est produit\\
    1 -- Plein jour\\
    2 -- Crépuscule ou aube\\
    3 -- Nuit sans éclairage public\\
    4 -- Nuit avec éclairage public non allumé\\
    5 -- Nuit avec éclairage public allumé
  \item
    Com : Le numéro de commune est un code donné par l`INSEE. Le code est composé du code INSEE du département suivi par 3 chiffres
  \item
    Agg : Localisation de l'accident\\
    1 -- Hors agglomération\\
    2 -- En agglomération
  \item
    Int : Le type d'intersection où l'accident s'est déroulé\\
    1 -- Hors intersection\\
    2 -- Intersection en X\\
    3 -- Intersection en T\\
    4 -- Intersection en Y\\
    5 -- Intersection à plus de 4 branches\\
    6 -- Giratoire\\
    7 -- Place\\
    8 -- Passage à niveau\\
    9 -- Autre intersection
  \item
    Atm : Le type de condition atmosphérique\\
    -1 -- Non renseigné\\
    1 -- Normale\\
    2 -- Pluie légère\\
    3 -- Pluie forte\\
    4 -- Neige - Grêle\\
    5 -- Brouillard - Fumée\\
    6 -- Vent fort - Tempête\\
    7 -- Temps éblouissant\\
    8 -- Temps couvert\\
    9 -- Couvert
  \item
    Col : Le type de collision\\
    -1 -- Non renseigné
    1 -- Deux véhicules - frontale\\
    2 -- Deux véhicules - par l'arrière\\
    3 -- Deux véhicules - par le côté\\
    4 -- Trois véhicules et plus - en chaîne\\
    5 -- Trois véhicules et plus - collisions multiples\\
    6 -- Autre collision\\
    7 -- Sans collision
  \end{itemize}
\item
  \textbf{Base lieux} :

  \begin{itemize}
  \item
    Catr : Catégorie de route\\
    1 -- Autoroute\\
    2 -- Route nationale\\
    3 -- Route départementale\\
    4 -- Voie communale\\
    5 -- Hors réseau public\\
    6 -- Parc de stationnement ouvert à la circulation publique\\
    7 -- Routes de métropole urbaine\\
    9 -- Autre
  \item
    Voie : Numéro de la route
  \item
    Circ : Régime de circulation\\
    -1 -- Non renseigné\\
    1 -- À sens unique\\
    2 -- Bidirectionnelle\\
    3 -- À chaussées séparées\\
    4 -- Avec voies d'affectation variable
  \item
    Nbv : Nombre total de voies de circulation
  \item
    Vosp : Signale l'existence d'une voie réservée, indépendamment du fait que l'accident ait lieu ou non sur cette voie\\
    -1 -- Non renseigné\\
    0 -- Sans objet\\
    1 -- Piste cyclable\\
    2 -- Bande cyclable\\
    3 -- Voie réservée
  \item
    Prof : Profil en long décrit la déclivité de la route à l'endroit de l'accident\\
    -1 -- Non renseigné\\
    1 -- Plat\\
    2 -- Pente\\
    3 -- Sommet de côte\\
    4 -- Bas de côte
  \item
    Plan : Tracé en plan\\
    -1 -- Non renseigné\\
    1 -- Partie rectiligne\\
    2 -- En courbe à gauche\\
    3 -- En courbe à droite\\
    4 -- En ``s''
  \item
    Larrout : Largeur de la chaussée affectée à la circulation des véhicules, ne sont pas compris les bandes d'arrêts d'urgences, les TPC et les places de stationnement (en m)
  \item
    Surf : État de la surface\\
    -1 -- Non renseigné\\
    1 -- Normale\\
    2 -- Mouillée\\
    3 -- Flaques\\
    4 -- Inondée\\
    5 -- Enneigée\\
    6 -- Boue\\
    7 -- Verglacée\\
    8 -- Corps gras - huile\\
    9 -- Autre
  \item
    Infra : Aménagement, infrastructure\\
    -1 -- Non renseigné\\
    0 -- Aucun\\
    1 -- Souterrain - tunnel\\
    2 -- Pont - autopont\\
    3 -- Bretelle d'échangeru ou de raccordement\\
    4 -- Voie ferrée\\
    5 -- Carrefour aménagé\\
    6 -- Zone piétonne\\
    7 -- Zone de péage\\
    8 -- Chantier\\
    9 -- Autres
  \item
    Situ : Situation de l'accident\\
    -1 -- Non renseigné\\
    0 -- Aucun\\
    1 -- Sur chaussée\\
    2 -- Sur bandes d'arrêt d'urgence\\
    3 -- Sur accotement\\
    4 -- Sur trottoir\\
    5 -- Sur piste cyclable\\
    6 -- Sur autre voie spéciale\\
    8 -- Autres
  \item
    VMA : Vitesse maximale autorisée sur le lieu et au moment de l'accident
  \end{itemize}
\item
  \textbf{Base véhicule} :

  \begin{itemize}
  \item
    Id\_vehicule : Identifiant unique du véhicule repris pour chacun des usagers occupant ce véhicule (y compris les piétons qui sont rattachés aux véhicules qui les ont heurtés) -- Code numérique
  \item
    Catv : Catégorie du véhicule :
    00 -- Indéterminable\\
    01 -- Bicyclette\\
    02 -- Cyclomoteur \textless50cm3\\
    03 -- Voiturette (Quadricycle à moteur carrossé) (anciennement ``voiturette ou tricycle à moteur'')\\
    04 -- Référence inutilisée depuis 2006 (scooter immatriculé)\\
    05 -- Référence inutilisée depuis 2006 (motocyclette)\\
    06 -- Référence inutilisée depuis 2006 (side-car)\\
    07 -- VL seul\\
    08 -- Référence inutilisée depuis 2006 (VL + caravane)\\
    09 -- Référence inutilisée depuis 2006 (VL + remorque)\\
    10 -- VU seul 1,5T \textless= PTAC \textless= 3,5T avec ou sans remorque (anciennement VU seul 1,5T \textless= PTAC
    \textless= 3,5T)\\
    11 -- Référence inutilisée depuis 2006 (VU (10) + caravane)\\
    12 -- Référence inutilisée depuis 2006 (VU (10) + remorque)\\
    13 -- PL seul 3,5T \textless PTCA \textless= 7,5T\\
    14 -- PL seul \textgreater{} 7,5T\\
    15 -- PL \textgreater{} 3,5T + remorque\\
    16 -- Tracteur routier seul\\
    17 -- Tracteur routier + semi-remorque\\
    18 -- Référence inutilisée depuis 2006 (transport en commun)\\
    19 -- Référence inutilisée depuis 2006 (tramway)\\
    20 -- Engin spécial\\
    21 -- Tracteur agricole\\
    30 -- Scooter \textless{} 50 cm3\\
    31 -- Motocyclette \textgreater{} 50 cm3 et \textless= 125 cm3\\
    32 -- Scooter \textgreater{} 50 cm3 et \textless= 125 cm3\\
    33 -- Motocyclette \textgreater{} 125 cm3\\
    34 -- Scooter \textgreater{} 125 cm3\\
    35 -- Quad léger \textless= 50 cm3 (Quadricycle à moteur non carrossé)\\
    36 -- Quad lourd \textgreater{} 50 cm3 (Quadricycle à moteur non carrossé)\\
    37 -- Autobus\\
    38 -- Autocar\\
    39 -- Train\\
    40 -- Tramway\\
    41 -- 3RM \textless= 50 cm3\\
    42 -- 3RM \textgreater{} 50 cm3 \textless= 125 cm3\\
    43 -- 3RM \textgreater{} 125 cm3\\
    50 -- EDP à moteur\\
    60 -- EDP sans moteur 80 -- VAE\\
    99 -- Autre véhicule
  \item
    Obs : Obstacle fixe heurté\\
    -1 -- Non renseigné\\
    0 -- Sans objet\\
    1 -- Véhicule en stationnement\\
    2 -- Arbre\\
    3 -- Glissière métallique\\
    4 -- Glissière béton\\
    5 -- Autre glissière\\
    6 -- Bâtiment, mur, pile de pont\\
    7 -- Support de signalisation verticale ou poste d'appel d'urgence\\
    8 -- Poteau\\
    9 -- Mobilier urbain\\
    10 -- Parapet\\
    11 -- Ilot, refuge, borne haute\\
    12 -- Bordure de trottoir\\
    13 -- Fossé, talus, paroi rocheuse\\
    14 -- Autre obstacle fixe sur chaussée\\
    15 -- Autre obstacle fixe sur trottoir ou accotement\\
    16 -- Sortie de chaussée sans obstacle\\
    17 -- Buse -- tête d'aqueduc
  \item
    Obsm : Obstacle mobile heurté\\
    -1 -- Non renseigné\\
    0 -- Aucun\\
    1 -- Piéton\\
    2 -- Véhicule\\
    4 -- Véhicule sur rail\\
    5 -- Animal domestique\\
    6 -- Animal sauvage\\
    9 -- Autre
  \item
    Manv : Manoeuvre principale avant l'accident\\
    -1 -- Non renseigné\\
    0 -- Inconnue\\
    1 -- Sans changement de direction\\
    2 -- Même sens, même file\\
    3 -- Entre 2 files\\
    4 -- En marche arrière\\
    5 -- A contresens\\
    6 -- En franchissant le terre-plein central\\
    7 -- Dans le couloir bus, dans le même sens\\
    8 -- Dans le couloir bus, dans le sens inverse\\
    9 -- En s'insérant
    10 -- En faisant demi-tour sur la chaussée\\
    \textbf{Changeant de file}
    11 -- A gauche\\
    12 -- A droite\\
    \textbf{Déporté}\\
    13 -- A gauche\\
    14 -- A droite\\
    \textbf{Tournant}\\
    15 -- A gauche\\
    16 -- A droite\\
    \textbf{Dépassant}\\
    17 -- A gauche\\
    18 -- A droite\\
    \textbf{Divers}\\
    19 -- Traversant la chaussée\\
    20 -- Manœuvre de stationnement\\
    21 -- Manœuvre d'évitement\\
    22 -- Ouverture de porte\\
    23 -- Arrêté (hors stationnement)\\
    24 -- En stationnement (avec occupants)\\
    25 -- Circulant sur trottoir\\
    26 -- Autres manœuvres
  \item
    Motor : Type de motorisation du véhicule\\
    -1 -- Non renseigné
    0 -- Inconnue\\
    1 -- Hydrocarbures\\
    2 -- Hybride électrique\\
    3 -- Electrique\\
    4 -- Hydrogène\\
    5 -- Humaine\\
    6 -- Autre
  \end{itemize}
\item
  \textbf{Base usagers} :

  \begin{itemize}
  \item
    Catu : Catégorie d'usager\\
    1 -- Conducteur\\
    2 -- Passager\\
    3 -- Piéton
  \item
    Grav: Gravité de blessure de l'usager, les usagers accidentés sont classés en trois catégories de victimes plus les indemnes\\
    1 -- Indemne\\
    2 -- Tué\\
    3 -- Blessé hospitalisé\\
    4 -- Blessé léger
  \item
    Sexe : Sexe de l'usager\\
    1 -- Masculin\\
    2 -- Féminin
  \item
    An\_nais : Année de naissance de l'usager
  \item
    Trajet : Motif du déplacement au moment de l'accident\\
    -1 -- Non renseigné\\
    0 -- Non renseigné\\
    1 -- Domicile -- travail\\
    2 -- Domicile -- école\\
    3 -- Courses -- achats\\
    4 -- Utilisation professionnelle\\
    5 -- Promenade -- loisirs\\
    9 -- Autre
  \item
    Secu1 :
    Le renseignement du caractère indique la présence et l'utilisation de l'équipement de sécurité\\
    -1 -- Non renseigné\\
    0 -- Aucun équipement\\
    1 -- Ceinture\\
    2 -- Casque\\
    3 -- Dispositif enfants\\
    4 -- Gilet réfléchissant\\
    5 -- Airbag (2RM/3RM)\\
    6 -- Gants (2RM/3RM)\\
    7 -- Gants + Airbag (2RM/3RM)\\
    8 -- Non déterminable\\
    9 -- Autre
  \item
    Secu2 : Le renseignement du caractère indique la présence et l'utilisation de l'équipement de sécurité\\
    -1 -- Non renseigné\\
    0 -- Aucun équipement\\
    1 -- Ceinture\\
    2 -- Casque\\
    3 -- Dispositif enfants\\
    4 -- Gilet réfléchissant\\
    5 -- Airbag (2RM/3RM)\\
    6 -- Gants (2RM/3RM)\\
    7 -- Gants + Airbag (2RM/3RM)\\
    8 -- Non déterminable\\
    9 -- Autre
  \item
    Secu3 : Le renseignement du caractère indique la présence et l'utilisation de l'équipement de sécurité\\
    -1 -- Non renseigné\\
    0 -- Aucun équipement\\
    1 -- Ceinture\\
    2 -- Casque\\
    3 -- Dispositif enfants\\
    4 -- Gilet réfléchissant\\
    5 -- Airbag (2RM/3RM)\\
    6 -- Gants (2RM/3RM)\\
    7 -- Gants + Airbag (2RM/3RM)\\
    8 -- Non déterminable\\
    9 -- Autre
  \item
    Locp : Localisation du piéton\\
    -1 -- Non renseigné\\
    0 -- Sans objet\\
    \textbf{Sur chaussée :}\\
    1 -- A + 50 m du passage piéton\\
    2 -- A -- 50 m du passage piéton\\
    \textbf{Sur passage piéton :}\\
    3 -- Sans signalisation lumineuse\\
    4 -- Avec signalisation lumineuse\\
    \textbf{Divers :}\\
    5 -- Sur trottoir\\
    6 -- Sur accotement\\
    7 -- Sur refuge ou BAU\\
    8 -- Sur contre allée\\
    9 -- Inconnue
  \item
    Actp : Action du piéton\\
    -1 -- Non renseigné\\
    \textbf{Se déplaçant}\\
    0 -- Non renseigné ou sans objet\\
    1 -- Sens véhicule heurtant\\
    2 -- Sens inverse du véhicule\\
    \textbf{Divers}\\
    3 -- Traversant\\
    4 -- Masqué\\
    5 -- Jouant -- courant\\
    6 -- Avec animal\\
    9 -- Autre\\
    A -- Monte/descend du véhicule\\
    B -- Inconnue
  \item
    Etatp : Cette variable permet de préciser si le piéton accidenté était seul ou non\\
    -1 -- Non renseigné\\
    1 -- Seul\\
    2 -- Accompagné\\
    3 -- En groupe
  \end{itemize}
\end{itemize}

\hypertarget{valeurs-manquantes}{%
\subsection{Valeurs manquantes}\label{valeurs-manquantes}}

\hypertarget{statistiques-descriptives}{%
\section{Statistiques descriptives}\label{statistiques-descriptives}}

\hypertarget{base-caractuxe9ristiques}{%
\subsection{Base caractéristiques}\label{base-caractuxe9ristiques}}

\hypertarget{base-lieux}{%
\subsection{Base lieux}\label{base-lieux}}

\hypertarget{base-vuxe9hicule}{%
\subsection{Base véhicule}\label{base-vuxe9hicule}}

\hypertarget{base-usagers}{%
\subsection{Base usagers}\label{base-usagers}}

\hypertarget{appendix-annexes}{%
\appendix}


\hypertarget{annexes}{%
\section{Annexes}\label{annexes}}

% % 
% % 
% % 
% 
% 
% 
% 

\end{document}
